\documentclass{article}
\usepackage[utf8]{inputenc}
\usepackage[french]{babel}
\usepackage[T1]{fontenc}
\usepackage{amsmath}
\usepackage{shuffle}
\usepackage{amsfonts}
\usepackage[margin=1in]{geometry}
\usepackage{graphicx}
\usepackage{tikz}
\usepackage[thinlines]{easytable}
\usepackage{titlesec}
\usepackage{listings}
\usepackage{amsmath}
\usepackage{amssymb}
\usepackage{tabularx}
\usepackage{pifont}
\usetikzlibrary{automata, positioning, arrows}
\newtheorem{theorem}{THEOREM}
\newtheorem{proof}{PROOF}
\usepackage{etoolbox}
\AtBeginEnvironment{align}{\setcounter{equation}{0}}

\begin{document}
\begin{titlepage}
    \begin{center}
        \textsc{\large UNIVERSITÉ DE MONTRÉAL}\\[3.5cm]
        
        \textsc{\large DEVOIR 7}\\[2.5cm]
       
        \textsc{\large PAR}\\
        \textsc{\large CHENGZONG JIANG (20122046)}\\
        \textsc{\large MICHAEL PLANTE (20182677)}\\
        \textsc{\large VANESSA THIBAULT-SOUCY (20126808)}\\
        \textsc{\large JAYDAN ALADRO (20152077)}\\
        \textsc{\large SOUKAINA BENABID (20148642)}\\[3.5cm]
        
        
        \textsc{\large BACCALAURÉAT EN INFORMATIQUE}\\
        \textsc{\large FACULTÉ DES ARTS ET SCIENCES}\\[3.5cm]
        
        \textsc{\large TRAVAIL PRÉSENTÉ À GENA HAHN}\\
        \textsc{\large DANS LE CADRE DU COURS IFT 2105}\\
        \textsc{\large INTRODUCTION À L'INFORMATIQUE THÉORIQUE}\\[3cm]
        
        \textsc{\large 05 AVRIL 2021}\\
    \end{center}
\end{titlepage}
\section*{Question 2}
\large On veut prouver que tout langage régulier est hors contexte.
\newline 
\newline Soit $L_1$ un langage régulier quelconque. Par définition, on sait qu'il existe un automate $M=(Q,\Sigma,\delta,s,F)$ tel que $L(M)=L_1$. 
\newline 
\newline Soit la grammaire hors contexte $G$ tel que : $G=(V,\Sigma,R,S)$
\newline $G$ possedera une variable $v$ pour chacun des états de l'automate. 
\newline On définit : 
\newline $V=\{R_i | q_i \in Q\}$
\newline $\delta(q_i,a)=q_j$
\newline $R_i \rightarrow aR_j$
\newline $\forall q_i \in F, R_i \rightarrow \epsilon$
\newline $S=R_0$
\newline 
\newline On a prouvé alors qu'on peut definir une grammaire hors contexte pour n'importe quel langage regulier. Donc tout langage régulier est hors contexte. CQFD.

\newpage
\section*{Question 3}
\large On cherche à donnez la forme normale de Chomsky de la grammaire suivante, etape par etape.
\newline 
\newline $S \rightarrow aBBBCD|BC|A|BAB$
\newline $A \rightarrow aAa|aba|Cbb$
\newline $B \rightarrow BB|b|\epsilon$
\newline $C \rightarrow S|BABAB$
\newline $E \rightarrow a|b|bb|aa$
\newline 
\newline Tout d'abord on remarque que la règle $E \rightarrow a|b|bb|aa$ est inaccessible, elle est donc inutile et on peut la laisser tomber.
\newline On constate egalement, que le terminal $S$ figure à droite dans la règle \newline $C \rightarrow S|BABAB$. On doit donc ajouter la règle suivante : $S' \rightarrow S$
\newline 
\newline On obtient donc : 
\newline $S' \rightarrow S$
\newline $S \rightarrow aBBBCD|BC|A|BAB$
\newline $A \rightarrow aAa|aba|Cbb$
\newline $B \rightarrow BB|b|\epsilon$
\newline $C \rightarrow S|BABAB$
\newline
\newline \textbf{*} Ensuite, éliminons les règles $\epsilon$ : $B \rightarrow \epsilon$ :
\newline On obtient :
\newline $S' \rightarrow S$
\newline $S \rightarrow aBBBCD|BC|A|BAB|aBBCD|aCD|C|A|BA|AB$ \textcolor{red}{Il ne manque pas aBCD? Et il y a 2 A -Jay}
\newline $A \rightarrow aAa|aba|Cbb$
\newline $B \rightarrow BB|b|B$
\newline $C \rightarrow S|BABAB|ABAB|BAAB|BABA|AAB|AA|ABA|BAA$
\newline
\newline \textbf{*} On élimine également les règles unitaires (de la forme $A \rightarrow B$) :
\newline On commence par éliminer : $B \rightarrow B$ : On obtient : 
\newline $S' \rightarrow S$
\newline $S \rightarrow aBBBCD|BC|A|BAB|aBBCD|aCD|C|A|BA|AB$
\newline $A \rightarrow aAa|aba|Cbb$
\newline $B \rightarrow BB|b$
\newline $C \rightarrow S|BABAB|ABAB|BAAB|BABA|AAB|AA|ABA|BAA$
\newline 
\newline On élimine ensuite  : $S \rightarrow A$ : On obtient : 
\newline $S' \rightarrow S$
\newline $S \rightarrow aBBBCD|BC|A|BAB|aBBCD|aCD|C|A|BA|AB|aAa|aba|Cbb$
\newline $A \rightarrow aAa|aba|Cbb$
\newline $B \rightarrow BB|b$
\newline $C \rightarrow S|BABAB|ABAB|BAAB|BABA|AAB|AA|ABA|BAA$
\newline 
\newline 
On elimine ensuite : $S \rightarrow C$ : On obtient : 
\newline $S' \rightarrow S$
\newline $S \rightarrow aBBBCD|BC|A|BAB|aBBCD|aCD|C|A|BA|AB|aAa|aba|Cbb|S|BABAB|ABAB|BAAB|\newline BABA|AAB|AA|ABA|BAA$
\newline $A \rightarrow aAa|aba|Cbb$
\newline $B \rightarrow BB|b$
\newline $C \rightarrow S|BABAB|ABAB|BAAB|BABA|AAB|AA|ABA|BAA$
\newline 
On elimine enfin : $C \rightarrow S$ : On obtient : 
\newline $S' \rightarrow S$
\newline $S \rightarrow aBBBCD|BC|A|BAB|aBBCD|aCD|C|A|BA|AB|aAa|aba|Cbb|S|BABAB|ABAB|BAAB|\newline BABA|AAB|AA|ABA|BAA$
\newline $A \rightarrow aAa|aba|Cbb$
\newline $B \rightarrow BB|b$
\newline $C \rightarrow S|BABAB|ABAB|BAAB|BABA|AAB|AA|ABA|BAA|\newline aBBBCD|BC|A|BAB|aBBCD|aCD|C|A|BA|AB|aAa|aba|Cbb|\newline S|BABAB|ABAB|BAAB|BABA|AAB|AA|ABA|BAA$
\newline
\newline \textbf{*} On continue, 
\\
\\
\\
\\
\\
\\

[Vanessa]\\
\\
Ici on met le blabla de la question... \\
\\
\newline $S \rightarrow aBBBCD|BC|A|BAB$
\newline $A \rightarrow aAa|aba|Cbb$
\newline $B \rightarrow BB|b|\epsilon$
\newline $C \rightarrow S|BABAB$
\newline $E \rightarrow a|b|bb|aa$
\\
\\
\\
-Étape 1 : on enlève les règles qui sont inutile.\\

On enlève la règle E car elle n'est jamais référer par les autre règle de la grammaire, et donc elle ne peut pas être atteint via la variable de départ S.\\
\newline $S \rightarrow aBBBCD|BC|A|BAB$
\newline $A \rightarrow aAa|aba|Cbb$
\newline $B \rightarrow BB|b|\epsilon$
\newline $C \rightarrow S|BABAB$


On enlève également le mot aBBBCD car comme nous n'avons pas de règle D, le mots ne peut pas être terminal.
\newline $S \rightarrow BC|A|BAB$
\newline $A \rightarrow aAa|aba|Cbb$
\newline $B \rightarrow BB|b|\epsilon$
\newline $C \rightarrow S|BABAB$
\\
\\
\\
-Étape 2 : Ajouter une variable initiale $S_0$ et de la règle associée.\\
\newline $S \rightarrow BC|A|BAB$
\newline $A \rightarrow aAa|aba|Cbb$
\newline $B \rightarrow BB|b|\epsilon$
\newline $C \rightarrow S|BABAB$
\newline $S_0 \rightarrow S$
\\
\\
\\
-Étape 3 : Enlever les règle de la forme $A \rightarrow \epsilon$.\\

- $B \rightarrow \epsilon$ :\\
\newline $S \rightarrow BC|A|BAB$
\newline $A \rightarrow aAa|aba|Cbb$
\newline $B \rightarrow BB|b$
\newline $C \rightarrow S|BABAB$
\newline $S_0 \rightarrow S$
\\

- $S \rightarrow \epsilon$ par l'entremise de $B$ :\\
\newline $S \rightarrow BC|A|BAB|AB|BA|C$
\newline $A \rightarrow aAa|aba|Cbb$
\newline $B \rightarrow BB|b$
\newline $C \rightarrow S|BABAB$
\newline $S_0 \rightarrow S$
\\

- $C \rightarrow \epsilon$ par l'entremise de $B$ :\\
\newline $S \rightarrow BC|A|BAB|AB|BA|C$
\newline $A \rightarrow aAa|aba|Cbb$
\newline $B \rightarrow BB|b$
\newline $C \rightarrow S|BABAB|ABAB|BAAB|BABA|ABA|BAA|AAB|AA$
\newline $S_0 \rightarrow S$
\\
\\
\\
-Étape 4 : Enlever les règles de la forme $A \rightarrow B$\\

- $S \rightarrow A$ :
\newline $S \rightarrow BC|BAB|AB|BA|aAa|aba|Cbb|C|$
\newline $A \rightarrow aAa|aba|Cbb$
\newline $B \rightarrow BB|b$
\newline $C \rightarrow S|BABAB|ABAB|BAAB|BABA|ABA|BAA|AAB|AA$
\newline $S_0 \rightarrow S$
\\

- $S \rightarrow C$ :
\newline $S \rightarrow BC|BAB|AB|BA|aAa|aba|Cbb|BABAB|ABAB|BAAB|BABA|ABA|BAA|AAB|AA$
\newline $A \rightarrow aAa|aba|Cbb$
\newline $B \rightarrow BB|b$
\newline $C \rightarrow S|BABAB|ABAB|BAAB|BABA|ABA|BAA|AAB|AA$
\newline $S_0 \rightarrow S$
\\

- $C \rightarrow S$ :
\newline $S \rightarrow BC|BAB|AB|BA|aAa|aba|Cbb|BABAB|ABAB|BAAB|BABA|ABA|BAA|AAB|AA$
\newline $A \rightarrow aAa|aba|Cbb$
\newline $B \rightarrow BB|b$
\newline $C \rightarrow BC|BAB|AB|BA|aAa|aba|Cbb|BABAB|ABAB|BAAB|BABA|ABA|BAA|AAB|AA$
\newline $S_0 \rightarrow S$
\\

- $S_0 \rightarrow S$ :
\newline $S \rightarrow BC|BAB|AB|BA|aAa|aba|Cbb|BABAB|ABAB|BAAB|BABA|ABA|BAA|AAB|AA$
\newline $A \rightarrow aAa|aba|Cbb$
\newline $B \rightarrow BB|b$
\newline $C \rightarrow BC|BAB|AB|BA|aAa|aba|Cbb|BABAB|ABAB|BAAB|BABA|ABA|BAA|AAB|AA$
\newline $S_0 \rightarrow BC|BAB|AB|BA|aAa|aba|Cbb|BABAB|ABAB|BAAB|BABA|ABA|BAA|AAB|AA$
\\
\\
\\
- Étape 5 : Enlever les règles de la forme $A \rightarrow u_1...u_k$ pour $k>2$.\\

- $S \rightarrow u_1...u_k$ pour $k>2$ :
\newline $S \rightarrow BC|BX_6|AB|BA|X_5X_7|X_5X_8|CX_{13}|BX_{10}|X_6X_6|X_{11}X_6|X_{11}X_{11}|AX_{11}|X_{11}A|AX_6|AA$
\newline $A \rightarrow aAa|aba|Cbb$
\newline $B \rightarrow BB|b$
\newline $C \rightarrow BC|BAB|AB|BA|aAa|aba|Cbb|BABAB|ABAB|BAAB|BABA|ABA|BAA|AAB|AA$
\newline $S_0 \rightarrow BC|BAB|AB|BA|aAa|aba|Cbb|BABAB|ABAB|BAAB|BABA|ABA|BAA|AAB|AA$
\newline $X_1 \rightarrow BX_2$
\newline $X_2 \rightarrow BX_3$
\newline $X_3 \rightarrow BC$
\newline $X_5 \rightarrow a$
\newline $X_6 \rightarrow AB$
\newline $X_7 \rightarrow AX_5$
\newline $X_8 \rightarrow ba$
\newline $X_9 \rightarrow bb$
\newline $X_{10} \rightarrow X_6X_6$
\newline $X_{11} \rightarrow BA$
\\
\\

- $A \rightarrow u_1...u_k$ pour $k>2$ :
\newline $S \rightarrow BC|BX_6|AB|BA|X_5X_7|X_5X_8|CX_{13}|BX_{10}|X_6X_6|X_{11}X_6|X_{11}X_{11}|AX_{11}|X_{11}A|AX_6|AA$
\newline $A \rightarrow X_5X_7|X_5X_8|CX_{13}$
\newline $B \rightarrow BB|b$
\newline $C \rightarrow BC|BAB|AB|BA|aAa|aba|Cbb|BABAB|ABAB|BAAB|BABA|ABA|BAA|AAB|AA$
\newline $S_0 \rightarrow BC|BAB|AB|BA|aAa|aba|Cbb|BABAB|ABAB|BAAB|BABA|ABA|BAA|AAB|AA$
\newline $X_1 \rightarrow BX_2$
\newline $X_2 \rightarrow BX_3$
\newline $X_3 \rightarrow BC$
\newline $X_5 \rightarrow a$
\newline $X_6 \rightarrow AB$
\newline $X_7 \rightarrow AX_5$
\newline $X_8 \rightarrow ba$
\newline $X_9 \rightarrow bb$
\newline $X_{10} \rightarrow X_6X_6$
\newline $X_{11} \rightarrow BA$
\\

- $C \rightarrow u_1...u_k$ pour $k>2$ :
\newline $S \rightarrow BC|BX_6|AB|BA|X_5X_7|X_5X_8|CX_{13}|BX_{10}|X_6X_6|X_{11}X_6|X_{11}X_{11}|AX_{11}|X_{11}A|AX_6|AA$
\newline $A \rightarrow X_5X_7|X_5X_8|CX_{13}$
\newline $B \rightarrow BB|b$
\newline $C \rightarrow BC|BX_6|AB|BA|X_5X_7|X_5X_8|CX_{13}|BX_{10}|X_6X_6|X_{11}X_6|X_{11}X_{11}|AX_{11}|X_{11}A|AX_6|AA$
\newline $S_0 \rightarrow BC|BAB|AB|BA|aAa|aba|Cbb|BABAB|ABAB|BAAB|BABA|ABA|BAA|AAB|AA$
\newline $X_1 \rightarrow BX_2$
\newline $X_2 \rightarrow BX_3$
\newline $X_3 \rightarrow BC$
\newline $X_5 \rightarrow a$
\newline $X_6 \rightarrow AB$
\newline $X_7 \rightarrow AX_5$
\newline $X_8 \rightarrow ba$
\newline $X_9 \rightarrow bb$
\newline $X_{10} \rightarrow X_6X_6$
\newline $X_{11} \rightarrow BA$
\\

- $S_0 \rightarrow u_1...u_k$ pour $k>2$ :
\newline $S \rightarrow BC|BX_6|AB|BA|X_5X_7|X_5X_8|CX_{13}|BX_{10}|X_6X_6|X_{11}X_6|X_{11}X_{11}|AX_{11}|X_{11}A|AX_6|AA$
\newline $A \rightarrow X_5X_7|X_5X_8|CX_{13}$
\newline $B \rightarrow BB|b$
\newline $C \rightarrow BC|BX_6|AB|BA|X_5X_7|X_5X_8|CX_{13}|BX_{10}|X_6X_6|X_{11}X_6|X_{11}X_{11}|AX_{11}|X_{11}A|AX_6|AA$
\newline $S_0 \rightarrow BC|BX_6|AB|BA|X_5X_7|X_5X_8|CX_{13}|BX_{10}|X_6X_6|X_{11}X_6|X_{11}X_{11}|AX_{11}|X_{11}A|AX_6|AA$
\newline $X_1 \rightarrow BX_2$
\newline $X_2 \rightarrow BX_3$
\newline $X_3 \rightarrow BC$
\newline $X_5 \rightarrow a$
\newline $X_6 \rightarrow AB$
\newline $X_7 \rightarrow AX_5$
\newline $X_8 \rightarrow ba$
\newline $X_9 \rightarrow bb$
\newline $X_{10} \rightarrow X_6X_6$
\newline $X_{11} \rightarrow BA$
\\
\\
\\
-Étape 6 : Enlever les règles de la forme $A \rightarrow xB$, $A \rightarrow Bx$ et $A \rightarrow xy$ et ajouter des règles qui génère les variables.
\\

Nous avons traiter tous les cas pour les regles de la forme $A \rightarrow xB$ et $A \rightarrow Bx$ lors de l'étape précédente.\\
\\

Traitons maintenant les règle de la forme : $A \rightarrow xy$\\

- pour les règles $X_8 \rightarrow ba$ et $X_9 \rightarrow bb$  :
\newline $S \rightarrow BC|BX_6|AB|BA|X_5X_7|X_5X_8|CX_{13}|BX_{10}|X_6X_6|X_{11}X_6|X_{11}X_{11}|AX_{11}|X_{11}A|AX_6|AA$
\newline $A \rightarrow X_5X_7|X_5X_8|CX_{13}$
\newline $B \rightarrow BB|b$
\newline $C \rightarrow BC|BX_6|AB|BA|X_5X_7|X_5X_8|CX_{13}|BX_{10}|X_6X_6|X_{11}X_6|X_{11}X_{11}|AX_{11}|X_{11}A|AX_6|AA$
\newline $S_0 \rightarrow BC|BX_6|AB|BA|X_5X_7|X_5X_8|CX_{13}|BX_{10}|X_6X_6|X_{11}X_6|X_{11}X_{11}|AX_{11}|X_{11}A|AX_6|AA$
\newline $X_1 \rightarrow BX_2$
\newline $X_2 \rightarrow BX_3$
\newline $X_3 \rightarrow BC$
\newline $X_5 \rightarrow a$
\newline $X_6 \rightarrow AB$
\newline $X_7 \rightarrow AX_5$
\newline $X_8 \rightarrow X'_1X_5$
\newline $X_9 \rightarrow X'_1X'_1$
\newline $X_{10} \rightarrow X_6X_6$
\newline $X_{11} \rightarrow BA$
\newline $X'_1 \rightarrow b$
\\
\\
Conclusion :\\
La grammaire Hors-context :
\newline $S \rightarrow aBBBCD|BC|A|BAB$
\newline $A \rightarrow aAa|aba|Cbb$
\newline $B \rightarrow BB|b|\epsilon$
\newline $C \rightarrow S|BABAB$
\newline $E \rightarrow a|b|bb|aa$

peut se réécrire sour la forme normale de Chomsky comme :\\
\newline $S \rightarrow BC|BX_6|AB|BA|X_5X_7|X_5X_8|CX_{13}|BX_{10}|X_6X_6|X_{11}X_6|X_{11}X_{11}|AX_{11}|X_{11}A|AX_6|AA$
\newline $A \rightarrow X_5X_7|X_5X_8|CX_{13}$
\newline $B \rightarrow BB|b$
\newline $C \rightarrow BC|BX_6|AB|BA|X_5X_7|X_5X_8|CX_{13}|BX_{10}|X_6X_6|X_{11}X_6|X_{11}X_{11}|AX_{11}|X_{11}A|AX_6|AA$
\newline $S_0 \rightarrow BC|BX_6|AB|BA|X_5X_7|X_5X_8|CX_{13}|BX_{10}|X_6X_6|X_{11}X_6|X_{11}X_{11}|AX_{11}|X_{11}A|AX_6|AA$
\newline $X_1 \rightarrow BX_2$
\newline $X_2 \rightarrow BX_3$
\newline $X_3 \rightarrow BC$

\newline $X_5 \rightarrow a$
\newline $X_6 \rightarrow AB$
\newline $X_7 \rightarrow AX_5$
\newline $X_8 \rightarrow X'_1X_5$
\newline $X_9 \rightarrow X'_1X'_1$
\newline $X_{10} \rightarrow X_6X_6$
\newline $X_{11} \rightarrow BA$
\newline $X'_1 \rightarrow b$
\\
\\
{Fin Vanessa}
\\
\\
(Jaydan)
\\
\\
J'ai fais comme vous mais quand on met de de la forme $A \rightarrow u_1...u_k\; pour\; k \geq 3$ j'y suis allé plus direct et j'ai aussi directement fais la supression des formes $A \rightarrow xy$. Je suppose il y a plusieurs possibilités, voici la mienne, très similaire à Vanessa:
\\
\newline $S_0$ apres les étapes précédentes
\begin{align*}
    S_0 \rightarrow &aBBC | BC | BAB |aBBC | aBC | aC | BA | AB | aAa | aba | Cbb | BABAB | AA | ABA | AAB | BAA \\
    &| BABA | ABAB | BAAB\\
\end{align*}
\begin{align*}
    S_0 \rightarrow &X5X1|BC|BX_9|X_5X_3|X_5X_4|BA|AB|X_5X_{11}|CX{13}|X_{10}X_{15}|AA|AX{10}|AX_{9}\\
    &|X_{10}A|X_{15}A|AX_{15}|X_9X_{10}\\
    S \rightarrow &X5X1|BC|BX_9|X_5X_3|X_5X_4|BA|AB|X_5X_{11}|CX{13}|X_{10}X_{15}|AA|AX{10}|AX_{9}\\
    &|X_{10}A|X_{15}A|AX_{15}|X_9X_{10}\\
    A \rightarrow &X_5X_7|X_5X_8|CX_{13}\\
    B \rightarrow &BB|b\\
    C \rightarrow &S\quad \mbox{On peut faire ca? Si oui on peut pas faire quelque chose pour les S?}\\
    \\
    X_1 \rightarrow &BX_2\quad  (BBBC)\\
    X_2 \rightarrow &BX_3\quad  (BBC)\\
    X_3 \rightarrow &BX_4\quad  (BC)\\
    X_4 \rightarrow &C\\
    X_5 \rightarrow &a\\
    X_6 \rightarrow &b\\
    X_7 \rightarrow &A\\
    X_8 \rightarrow &B\\
    X_9 \rightarrow &X_7X_8\quad  (AB)\\
    X_{10} \rightarrow &X_8X_7\quad  (BA)\\
    X_{11} \rightarrow &X_7X_5\quad  (Aa)\\
    X_{12} \rightarrow &X_6X_5\quad  (ba)\\
    X_{13} \rightarrow &X_6X_6\quad  (bb)\\
    X_{14} \rightarrow &X_{10}X_8\quad  (BAB)\\
\end{align*}
\\
\\
Aussi, si vous regardez l'avant derniere démo, Samuel insiste beaucoup sur le fait d'enlever les variables inutile et qu'il faut un peu faire de l'effort pour (hint?)
\\
Une question, si on enleve D, ce n'est pas illegal de mettre $X_4 \rightarrow &CD$ par exemple? Vu que bah ca n'existe pas
\end{document}


